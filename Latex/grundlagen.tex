\section{Problemstellung}
Wie in der Einleitung bereits angemerkt, beschreibt \glqq Information Retrieval\grqq das Bereitstellen spezieller Informationen aus einer meist großen, unsortierten Datenmenge. Dabei bekommt das System eine vom Nutzer gestellte Query (Abfrage) und versucht auf dessen Basis, Daten, die meist als Dokumente vorliegen, zurückzuliefern.
Im Gegensatz zu Abfragen im Datenbankumfeld beinhaltet die Query jedoch keinerlei Informationen, um ein spezielles Element eindeutig identifizieren zu können. Dies soll ein IR-System auch nicht leisten. Vielmehr sollen Ergebnisse zurückgeliefert werden, die mit hoher Wahrscheinlichkeit Relevanz bzgl. der gestellten Query besitzen. Der Nutzer selektiert dann die für diesen nötigen Dokumente.
\newline
Mathematisch lässt sich dies folgendermaßen formulieren:
Aus einer Dokumentenmenge $D$ soll mithilfe einer Funktion eine Teilmenge $D_1$ von $D$ ermittelt werden, die relevant für eine Abfrage $q$ ist.
\newline
Um diese Funktion sinnvoll definieren zu können, muss jedoch zuvor die Menge aller Queries, sowie die Menge aller Tokens definiert werden:
\begin{defi}
	Sei $d$ \in $D$ ein Dokument. Die Menge $T_d$ ist nun die Menge aller Wörter, die in dem Dokument $d$ enthalten sind: $T_d$ = $\{$$t_1$, .., $t_n$$\}$.
	\newline
	Die Menge $T$ ist die Menge aller Terme, die in den Dokumenten aus $D$ vorkommen, also:
	$T$ = $T_{d1}$ \cup .. \cup $T_{dn}$ mit $d_i$ \in $D$.
\end{defi}

\begin{defi}
	Mithilfe der letzten Definition kann nun die Menge aller möglichen Queries definiert werden:
	$Q$ \subseteq $2^T$
\end{defi}

\begin{defi}
	Eine Funktion $f$: $Q$ \rightarrow $D_1$ heißt Retrievalfunktion, wobei $D_1$ \subseteq $D$ gilt und $Q$ die Menge aller Queries ist.
\end{defi}

Nachdem die Problemstellung formuliert ist, muss eine Strategie entwickelt werden, wie die Funktion $f$ dargestellt bzw. umgesetzt werden kann.

\newpage

\section{Strategiefindung}
Dieser Abschnitt soll eine Übersicht bieten, wie das im Folgenden vorgestellte IR-System arbeiten soll.
\newline
Als Vorarbeit müssen alle Dokumente, die im Index aufgenommen werden sollen, in eine Codierung wie ASCII oder Unicode umgewandelt werden. Dazu wird ein Tool genutzt, das hier nicht weiter von Relevanz sein wird.
Es sollen mindestens all diejenigen Dokumente in den Index aufgenommen werden, die im PDF-Format vorliegen.
\newline
Der erste Schritt, der das IR-System an sich leisten muss, ist das Erstellen von Tokens. Dazu wird jedes Dokument in Tokens aufgespalten. Ein Token ist in den meisten Fällen ein Wort, Satzzeichen wie Leerzeichen, Kommata usw. sollen nicht als Tokens behandelt werden und werden ignoriert.
\newline
Für jeden Token wird es später im Index einen Eintrag geben, der eine Liste mit weiteren Informationen hält. Diese Liste muss mindestens die Dokument-ID speichern, in dem das Token steht. In diesen Listen werden häufig noch weitere Informationen hinterlegt, beispielsweise die Häufigkeit eines Tokens.
\newline
Der zweite große Schritt besteht darin, einen Algorithmus zu entwerfen, der eine Query entgegennimmt und auf Basis der Query und des Index eine Liste von relevanten Dokumenten ausgibt. Dieser wird das in der Einleitung kurz vorgestellte Vektorraummodell verwenden. Weiter wird dieser für die Ermittlung der Relevanz die sogenannte TF-IDF-Gewichtung nutzen. Diese wird später noch ausführlich vorgestellt.
\newline
Neben diesen beiden Punkten wird der Index komprimiert, um Speicherplatz zu sparen und die Performance zu erhöhen.