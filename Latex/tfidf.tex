Bei großen Sammlungen von Dokumenten reicht es nicht aus, wenn das IR die Dokumente, welche die Suchanfrage erfüllen, zurück gibt. Die Menge der Dokumente, die durch das IR zurückgegebene werden, ist meist so groß, dass der Nutzer nicht in der Lage ist alle Dokumente zu sichten und die für ihn relevanten Dokument auszuwählen.\\
Die Frage, die sich hier stellt, ist, wie der Nutzer die Dokumente bekommt, die er mit hoher wahrscheinlichsten benötigt. Hier fällt der Begriff des Scorings. Scoring bedeutet so viel wie Bewertung und wird genutzt um zu bestimmen, welche Dokumente für die Suchanfrage am relevantesten sind. Es kann auch von einer Gewichtung der Dokumente gesprochen werden.\\
Das Gewichtungsmodell, welches in dieser Arbeit thematisiert und genutzt wird ist das TF-IDF-Modell. TF steht hierbei für Term Frequency und IDF für Inverse Document Frequency. Beide Methoden werde einzeln in den folgenden Abschnitten vorgestellt und am Ende verknüpft.

\section{Term Frequency}

\section{Inverse Document Frequency}